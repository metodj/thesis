\documentclass[12pt,a4paper]{amsart}
% ukazi za delo s slovenscino -- izberi kodiranje, ki ti ustreza
\usepackage[slovene]{babel}
%\usepackage[cp1250]{inputenc}
%\usepackage[T1]{fontenc}
\usepackage[utf8]{inputenc}
\usepackage{amsmath,amssymb,amsfonts}
\usepackage{url}
%\usepackage[normalem]{ulem}
\usepackage[dvipsnames,usenames]{color}

\usepackage{hyperref}

% ne spreminjaj podatkov, ki vplivajo na obliko strani
\textwidth 15cm
\textheight 24cm
\oddsidemargin.5cm
\evensidemargin.5cm
\topmargin-5mm
\addtolength{\footskip}{10pt}
\pagestyle{plain}
%\overfullrule=15pt % oznaci predlogo vrstico
\overfullrule=0pt


% ukazi za matematicna okolja
\theoremstyle{definition} % tekst napisan pokoncno
\newtheorem{definicija}{Definicija}[section]
\newtheorem{primer}[definicija]{Primer}
\newtheorem{opomba}[definicija]{Opomba}

\renewcommand\endprimer{\hfill$\diamondsuit$}


\theoremstyle{plain} % tekst napisan posevno
\newtheorem{lema}[definicija]{Lema}
\newtheorem{izrek}[definicija]{Izrek}
\newtheorem{trditev}[definicija]{Trditev}
\newtheorem{posledica}[definicija]{Posledica}


% za stevilske mnozice uporabi naslednje simbole
\newcommand{\R}{\mathbb R}
\newcommand{\N}{\mathbb N}
\newcommand{\Z}{\mathbb Z}
\newcommand{\C}{\mathbb C}
\newcommand{\Q}{\mathbb Q}

% ukaz za slovarsko geslo
\newlength{\odstavek}
\setlength{\odstavek}{\parindent}
\newcommand{\geslo}[2]{\noindent\textbf{#1}\hspace*{3mm}\hangindent=\parindent\hangafter=1 #2}

% naslednje ukaze ustrezno popravi
\newcommand{\program}{Finančna matematika} % ime studijskega programa: Matematika/Finan"cna matematika
\newcommand{\imeavtorja}{Metod Jazbec} % ime avtorja
\newcommand{\imementorja}{doc. dr. Aljoša Peperko} % akademski naziv in ime mentorja
\newcommand{\naslovdela}{Splošna definicija diferencirane zasebnosti}
\newcommand{\letnica}{2018} %letnica diplome


% vstavi svoje definicije ...




\begin{document}
Vprašanja se navezujejo na članek Differential Privacy for Functions and Functional Data: \url{http://www.jmlr.org/papers/volume14/hall13a/hall13a.pdf}
\newline
\begin{itemize}
\item Proposition 7 (v diplomi izrek 6.1): kaj nam zagotavlja pozitivno definitnost Grammove matrike $M$? To da je $K$ kovariančna funkcija Gaussovega procesa, nam zagotavlja samo pozitivno semi-definitnost. Pozitivno definitnost pa verjetno potrebujemo, saj se v dokazu skličemo na Proposition 3 (v diplomi izrek 6.2), kjer je pozitivna definitnost matrike M ena od predpostavk izreka. Vprašanje na to temo sem zastavil tudi na matematičnem forumu: \url{https://stats.stackexchange.com/questions/357969/positive-definiteness-of-grammian-with-respect-to-gaussian-process-covariance-f/357981?noredirect=1#comment673146_357981}
\newline
\newline
Tudi ali je M sploh Grammova matrika. K(x,y) namreč ni skalarni produkt... V nadaljevanju, ko delamo z RKHS, postane predstavitev s skalarnim produktom logična, saj imamo $\langle K_x, K_y \rangle_{\mathcal{H}} = K(x,y)$, na tem mestu pa mi poimenovanje matrike M kot Grammove matrike ni najbolj logično.
\newline
\item Proposition 8  (v diplomi izrek 6.8): obrnljivost matrike M, ki naj bi sledila iz Mercerjevega izreka, mi še vedno ni jasna. Kar sem sicer razmišljal je to, da iz lastnosti Grammove matrike sledi 
\begin{gather*}
M(x_1,...,x_n) = 
 \begin{pmatrix}
  K(x_1,x_1) & \cdots & K(x_1,x_n) \\
  \vdots    & \ddots & \vdots  \\
  K(x_n,x_1) & \cdots & K(x_n,x_n) 
 \end{pmatrix} \\ = 
  \begin{pmatrix}
  K_{x_1} & \cdots & K_{x_1} \\
  \vdots    & \ddots & \vdots  \\
  K_{x_n} & \cdots & K_{x_n} 
 \end{pmatrix}^T 
  \begin{pmatrix}
  K_{x_1} & \cdots & K_{x_1} \\
  \vdots    & \ddots & \vdots  \\
  K_{x_n} & \cdots & K_{x_n} 
 \end{pmatrix}.
\end{gather*}
Če matriki na desni strani označimo z $A^T$ in $A$, velja $\det{M} = det {A^2}$. $\det{A}$ pa bo različna od 0, natanko tedaj, ko bodo stolpci ali vrstice linearno neodvisni, to pa bo natanko tedaj, ko bodo funkcije $K_{x_i}$ med samo linearno neodvisne. Lahko iz tega, da so $x_i$ različne točke (predpostavke izreka), sklepamo na linearno neodvisnost funkcij $K_{x_i}$? 
\newline
\item Proposition 8 (v diplomi izrek 6.8): Ne vidim enakosti 
$$\langle f, Pf\rangle_{\mathcal{H}} = \|M^{-1/2}(x_1,...x_n)
\begin{pmatrix}
  f(x_1)  \\
  \vdots     \\
  f(x_n)
 \end{pmatrix}
\|_2^2  
$$
Sem si razpisal na dolgo za matriko dimenzije $2 \times 2$,  vendar vseeno nisem dobil enakega rezultata na obeh straneh enačaja.
\end{itemize}
\end{document}