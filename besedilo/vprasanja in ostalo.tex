\documentclass[a4paper]{article}
\usepackage[slovene]{babel}
\usepackage[utf8]{inputenc}
\usepackage[T1]{fontenc}
\usepackage{lmodern}
\usepackage{amssymb}
\usepackage{amsthm}
\usepackage{amsmath}
\usepackage{hyperref}

\theoremstyle{definition}
\newtheorem{definition}{Definicija}

\begin{document}
\textit{Differetnial privacy: mathematical framework for analysing \textbf{privacy-preserving data publishing and mining}}


\section{Vprašanja}
\begin{itemize}
\item $\sigma$-algebri v primerih 1 in 2.a. Sem jih uredu definiral?
\item Primer 6. A je tu v ozadju ubistvu Dynkinova lema? A je izrek o monotonih razredih ekvivalenten Dynkinovi lemi?
\item Primer 6.  $ \pi _{t_1,\ldots,t_k}: C([0,1]) \rightarrow \mathbb{R}^k$ Kako pokazati, da so te preslikave merljive?
\item Opomba nad izrekom 3. Ali prav razumem eno-dimenzionalne mehanizme (perturbacije podatkovne baze)?
\item Primer 9. Predpostavimo samo, da je $D$ omejen. A ne rabi biti še zaprt, da dobimo kompaktnost?  Tudi ali res sledi, da je podan mehanizem zaseben za vsako poizvedbo? V članku na koncu namreč navedejo \textit{this example can be seen as an application of the Laplacian mechanism to the identity query}. Verjetno res velja, samo za identity query in ne za vse, saj je Laplacov mehanizem pomojem oblike perturbacije odgovorov na poizvedbe. Izrek o Identity query pa dela z mehanizmi, ki perturbirajo podatkovno bazo.
\item Izrek 5. Ne razumem dela dokaza, kjer trdimo, da mora obstaja tak $v \neq u$, da velja $\mathbb{P}(X_u \in B_{t\gamma}(v)) \leq \frac{\gamma}{\kappa m}.$ 
\end{itemize}

\section{Predlogi za dopolnitev teoretičnega dela}
Lahko dodam še Example 6 in Example 7 iz izhodiščnega članka.
\newline
\newline
Pri izreku o monotonih razredih lahko dodam, da je to ekvivalentno Dynkinovi lemi in razložim zakaj (Bernik swag).
\newline
\url{https://math.stackexchange.com/questions/1193970/monotone-class-theorem-vs-dynkin-pi-lambda-theorem}
\newline
\newline
Pri definiciji diferencirane zasebnosti lahko pod opombe dodam še naslednji alineji:
\begin{itemize}
\item zakaj $e^\epsilon$ in ne $(1+\epsilon)$; computational convenience 
\item morda omenim in pojasnim katero od naslednjih lastnosti: composability, group privacy, robustness to auxiliary information (glej uvodni del drugega članka) 
\end{itemize}



\end{document}